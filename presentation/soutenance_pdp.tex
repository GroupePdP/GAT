\documentclass[8pt]{beamer}
\usepackage[utf8]{inputenc}
\usepackage{lmodern}
%\usepackage{subfig}
\usepackage{caption}

\setbeamertemplate{navigation symbols}{}

\usetheme{Warsaw}
\usecolortheme[RGB={32,69,96}]{structure}
\beamersetuncovermixins{\opaqueness<1>{75}}{\opaqueness<2->{55}}
\addtobeamertemplate{footline}{\hfill\insertframenumber/\inserttotalframenumber\hspace{2em}\null}
%\beamertemplatetransparentcovered


\newcommand {\framedgraphic}[2] {
    \begin{frame}{#1}
        \begin{center}
            \includegraphics[width=\textwidth,height=0.8\textheight,keepaspectratio]{#2}
        \end{center}
    \end{frame}
}

\title{Soutenance projet de programmation}
\subtitle{Génération Automatique de Texte}
\author{B. Barthés \\ A. Boumera \\ B. Guiomar \\ C. Pennarun \\ A. Wintringer}
\date{Mardi 16 Avril 2013}

\begin{document} 
\begin{frame}
\titlepage
\end{frame}

\section*{Introduction}
\begin{frame}\frametitle{Introduction}
\end{frame}


\section{Déroulement du logiciel}

\subsection{Administrateur}

\subsubsection{Base de données}
\begin{frame}{Base de données}
\begin{block}{Création de projet}
\begin{center}
\includegraphics[scale=0.2]{db_connection.png}
\end{center}
\end{block}
\end{frame}

\begin{frame}{Base de données}
\begin{block}{Organisation de la base de données}
\begin{center}
\includegraphics[scale=0.3]{db_presentation.png}
\end{center}
\end{block}
\end{frame}

\subsubsection{Base de données}
\begin{frame}{Base de données}
\begin{block}{Création de types}
\begin{center}
\includegraphics[scale=0.2]{creation_types.png}
\end{center}
\end{block}
\end{frame}


\subsubsection{Concepts et système de typage}
\begin{frame}{Concepts et système de typage}
\begin{columns}

\begin{column}{6cm}
\setbeamercovered{transparent}
\begin{itemize}[<+->]
\item Types
\\ \only<1>{Blabla types}
\item Concepts
\\ \only<2>{Blabla concepts}
\item Scénarios
\\ \only<3>{Blabla scénarios}
\end{itemize}
\end{column}

\begin{column}{6cm}
\only<1>{\includegraphics[scale=0.135]{creation_types.png}
\captionof{figure}[]{Fenêtre création de types}}

\only<2>{\includegraphics[scale=0.135]{creation_concepts.png}
\captionof{figure}[]{Fenêtre création de concepts}}

\only<3>{\includegraphics[scale=0.135]{creation_scenario.png}
\captionof{figure}[]{Fenêtre création de scénarios}}
\end{column}

\end{columns}
\end{frame}





\subsubsection{Sauvegarde}
\begin{frame}{Sauvegarde}
\end{frame}


\subsection{Utilisateur}

\subsubsection{Graphes conceptuels}
\begin{frame}{Graphes conceptuels}

\only<1-4>{
\uncover<1-4>{Point technique : rajout d'un noeud dans le graphe conceptuel}

\uncover<2-4>{\bigskip Problème : besoin d'une structure arborescente}
\uncover<3-4>{\\ MAIS plusieurs noeuds peuvent référencer le même Concept}

\uncover<4>{\bigskip Solution proposée : 
	\begin{itemize}
	\item utilisation d'un tag
	\item création d'une référence
	\end{itemize}
}
}

\only<5->{
\uncover<5->{addChild(GraphNode child, int index) sur un GraphNode parent}
\begin{itemize} 
\item \uncover<6->{on vérifie la compatibilité de child et parent}
\item \uncover<7->{si le noeud child est tagué}
				\uncover<8->{\\ on en crée une copie, qui en contient une référence}
				\uncover<9->{\\ on signale que child possède une référence}
\item  \uncover<10->{sinon, on met un tag dessus et on l'ajoute}
\end{itemize}

\uncover<11->{\bigskip Remarque : pas besoin de regarder les "enfants" du noeud à rajouter}
}
\end{frame}


\subsubsection{Connexion à Syntox}
\begin{frame}{Connexion à Syntox}
\begin{block}{Fonctionnement souhaité}
\begin{center}
\includegraphics[scale=0.35]{syntox_connection.jpg}
\end{center}
Problème technique: \\ Le texte généré par Syntox n'était pas obtenu, l'action du formulaire ne semblait pas se déclencher. \\
\end{block}
\end{frame}

\begin{frame}{Connexion à Syntox}
\begin{block}{Étude d'un nouveau procédé}
Rappel des besoins:
\begin{itemize}
\item communiquer avec Syntox,
\item lui fournir la requête générée par le GraphConcept,
\item récupérer et afficher le résultat.
\end{itemize}
Idée: déclencher de manière sûr l'action du formulaire. 
\end{block}
\end{frame}

\begin{frame}{Connexion à Syntox}
\begin{block}{Solution élaborée}
Création de la page HTML "pdp.html" en local.
\begin{center}
\includegraphics[scale=0.35]{syntox_connection2.png}
\end{center}
Les champs masquées (Grammaire, lexique et élément d'éditions) de la page originale en ligne sont transformés en fichier.
\\ Une fois générée, la page est automatique lancée depuis le navigateur par défaut de l'utilisateur.
\end{block}
\end{frame}

\begin{frame}{Connexion à Syntox}
\begin{block}{Avantages et inconvénients de cette solution}
Avantages:
\begin{itemize}
\item page en local, non-nécessité d'avoir internet pour la créer,
\item la requête peut être modifiée avant d'être lancé à Syntox,
\item les fichiers, grammaire, lexique et éléments d'éditions peuvent être édités.
\end{itemize}
Inconvénients:
\begin{itemize}
\item l'utilisateur doit lui même appuyer sur le bouton pour générer l'action du formulaire,
\item la mise à jour des fichiers doit se faire manuellement,
\item un navigateur internet désigné par défaut doit être présent.
\end{itemize}
\end{block}
\end{frame}

\section{Architecture}



\section{Test}
\begin{frame}
\end{frame}


\begin{frame}{Conclusion}
\section*{Conclusion}
\end{frame}


\end{document}
 
